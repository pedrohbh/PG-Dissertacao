% ==============================================================================
% Modelo para Monografia de Projeto de Graduação (PG)
% Prof. Vítor E. Silva Souza - Nemo / DI / UFES
%
% Baseado em abtex2-modelo-trabalho-academico.tex, v-1.9.2 laurocesar
% Copyright 2012-2014 by abnTeX2 group at http://abntex2.googlecode.com/ 
%
% This work may be distributed and/or modified under the conditions of the LaTeX 
% Project Public License, either version 1.3 of this license or (at your option) 
% any later version. The latest version of this license is in
% http://www.latex-project.org/lppl.txt.
%
% IMPORTANTE:
% Instruções encontram-se espalhadas pelo documento. Para facilitar sua leitura,
% tais instruções são precedidas por (*) -- utilize a função localizar do seu
% editor para passar por todas elas.
% ==============================================================================

% Usa o estilo abntex2, configurando detalhes de formatação e hifenização.
\documentclass[
	12pt,				% Tamanho da fonte.
	openright,			% Capítulos começam em página ímpar (insere página vazia caso preciso).
	oneside,			% Para impressão em verso e anverso. Oposto a oneside.
	a4paper,			% Tamanho do papel.
	english,			% Idioma adicional para hifenização.
	french,				% Idioma adicional para hifenização.
	spanish,			% Idioma adicional para hifenização.
	brazil				% O último idioma é o principal do documento.
	]{abntex2}



%%% Importação de pacotes. %%%
\usepackage{float}
\usepackage{adjustbox}
\usepackage{multirow}
% Conserta o erro "No room for a new \count"
\usepackage{etex}
%\reserveinserts{28}

% Usa a fonte Latin Modern.
\usepackage{lmodern}

% Seleção de códigos de fonte.
\usepackage[T1]{fontenc}

% Codificação do documento em Unicode.
\usepackage[utf8]{inputenc}

% Usado pela ficha catalográfica.
\usepackage{lastpage}

% Indenta o primeiro parágrafo de cada seção.
\usepackage{indentfirst}

% Controle das cores.
\usepackage[usenames,dvipsnames]{xcolor}

% Inclusão de gráficos.
\usepackage{graphicx}

% Inclusão de páginas em PDF diretamente no documento (para uso nos apêndices).
\usepackage{pdfpages}

% Para melhorias de justificação.
\usepackage{microtype}

% Citações padrão ABNT.
\usepackage[brazilian,hyperpageref]{backref}
\usepackage[alf]{abntex2cite}	
\renewcommand{\backrefpagesname}{Citado na(s) página(s):~}		% Usado sem a opção hyperpageref de backref.
\renewcommand{\backref}{}										% Texto padrão antes do número das páginas.
\renewcommand*{\backrefalt}[4]{									% Define os textos da citação.
	\ifcase #1
		Nenhuma citação no texto.
	\or
		Citado na página #2.
	\else
		Citado #1 vezes nas páginas #2.
	\fi}

% \rm is deprecated and should not be used in a LaTeX2e document
% http://tex.stackexchange.com/questions/151897/always-textrm-never-rm-a-counterexample
\renewcommand{\rm}{\textrm}

% Pacotes não incluídos no template abntex2. 
% Podem ser comentados caso não queira utilizá-los.

% Inclusão de símbolos não padrão.
\usepackage{amssymb}
\usepackage{eurosym}

% Para utilizar \eqref para referenciar equações.
\usepackage{amsmath}

% Permite mostrar figuras muito largas em modo paisagem com \begin{sidewaysfigure} ao invés de \begin{figure}.
\usepackage{rotating}

% Permite customizar listas enumeradas/com marcadores.
\usepackage{enumitem}

% Permite inserir hiperlinks com \url{}.
\usepackage{bigfoot}
\usepackage{hyperref}

% Permite usar o comando \hl{} para evidenciar texto com fundo amarelo. Útil para chamar atenção a itens a fazer.
\usepackage{soulutf8}

% Colorinlistoftodos package: to insert colored comments so authors can collaborate on the content.
\usepackage[colorinlistoftodos, textwidth=20mm, textsize=footnotesize]{todonotes}
\newcommand{\aluno}[1]{\todo[author=\textbf{Aluno},color=green!30,caption={},inline]{#1}}
\newcommand{\professor}[1]{\todo[author=\textbf{Professor},color=red!30,caption={},inline]{#1}}

% Permite inserir espaço em branco condicional (incluído no texto final só se necessário) em macros.
\usepackage{xspace}

% Permite incluir listagens de código com o comando \lstinputlisting{}.
\usepackage{listings}
\usepackage{caption}
\DeclareCaptionFont{white}{\color{white}}
\DeclareCaptionFormat{listing}{\colorbox{gray}{\parbox{\textwidth}{#1#2#3}}}
\captionsetup[lstlisting]{format=listing,labelfont=white,textfont=white}
\renewcommand{\lstlistingname}{Algoritmo}
\definecolor{mygray}{rgb}{0.5,0.5,0.5}
\lstset{
	basicstyle=\scriptsize,
	breaklines=true,
	numbers=left,
	numbersep=5pt,
	numberstyle=\tiny\color{mygray}, 
	rulecolor=\color{black},
	showstringspaces=false,
	tabsize=2,
    inputencoding=utf8,
    extendedchars=true,
    literate=%
    {é}{{\'{e}}}1
    {è}{{\`{e}}}1
    {ê}{{\^{e}}}1
    {ë}{{\¨{e}}}1
    {É}{{\'{E}}}1
    {Ê}{{\^{E}}}1
    {û}{{\^{u}}}1
    {ù}{{\`{u}}}1
    {â}{{\^{a}}}1
    {à}{{\`{a}}}1
    {á}{{\'{a}}}1
    {ã}{{\~{a}}}1
    {Á}{{\'{A}}}1
    {Â}{{\^{A}}}1
    {Ã}{{\~{A}}}1
    {ç}{{\c{c}}}1
    {Ç}{{\c{C}}}1
    {õ}{{\~{o}}}1
    {ó}{{\'{o}}}1
    {ô}{{\^{o}}}1
    {Õ}{{\~{O}}}1
    {Ó}{{\'{O}}}1
    {Ô}{{\^{O}}}1
    {î}{{\^{i}}}1
    {Î}{{\^{I}}}1
    {í}{{\'{i}}}1
    {Í}{{\~{Í}}}1
}



%%% Definição de variáveis. %%%

% (*) Substituir os textos abaixo com as informações apropriadas.
\titulo{Estudo sobre Algoritmos de Menor Caminho em Grafos}
\autor{Pedro}
\local{Vitória, ES}
\data{2017}
\orientador{Prof. Dra. Maria Claudia Silva Boeres}
%\coorientador{Nome do Co-orientador}
\instituicao{
  Universidade Federal do Espírito Santo -- UFES
  \par
  Centro Tecnológico
  \par
  Departamento de Informática}
\tipotrabalho{Monografia (PG)}

% Preâmbulo (tipo do trabalho, objetivo, nome da instituição, área de concentração, etc.).
% (*) Verificar se está correto (ex.: substituir por Engenharia de Computação se for o caso).
\preambulo{Monografia apresentada ao Curso de Ciência da Computação do Departamento de Informática da Universidade Federal do Espírito Santo, como requisito parcial para obtenção do Grau de Bacharel em Ciência da Computação.}

% Macros específicas do trabalho.
% (*) Inclua aqui termos que são utilizados muitas vezes e que demandam formatação especial.
% Os exemplos abaixo incluem i* (substituindo o asterisco por uma estrela) e Java com TM em superscript.
% Use sempre \xspace para que o LaTeX inclua espaço em branco após a macro somente quando necessário.
\newcommand{\istar}{\textit{i}$^\star$\xspace}
\newcommand{\java}{Java\texttrademark\xspace}
\newcommand{\latex}{\LaTeX\xspace}




%%% Configurações finais de aparência. %%%

% Altera o aspecto da cor azul.
\definecolor{blue}{RGB}{41,5,195}

% Informações do PDF.
\makeatletter
\hypersetup{
	pdftitle={\@title}, 
	pdfauthor={\@author},
	pdfsubject={\imprimirpreambulo},
	pdfcreator={LaTeX with abnTeX2},
	pdfkeywords={abnt}{latex}{abntex}{abntex2}{trabalho acadêmico}, 
	colorlinks=true,				% Colore os links (ao invés de usar caixas).
	linkcolor=blue,					% Cor dos links.
	citecolor=blue,					% Cor dos links na bibliografia.
	filecolor=magenta,				% Cor dos links de arquivo.
	urlcolor=blue,					% Cor das URLs.
	bookmarksdepth=4
}
\makeatother

% Espaçamentos entre linhas e parágrafos.
\setlength{\parindent}{1.3cm}
\setlength{\parskip}{0.2cm}



%%% Páginas iniciais do documento: capa, folha de rosto, ficha, resumo, tabelas, etc. %%%

% Compila o índice.
\makeindex

% Inicia o documento.
\begin{document}

% Retira espaço extra obsoleto entre as frases.
\frenchspacing

% Capa do trabalho.
\imprimircapa

% Folha de rosto (o * indica que haverá a ficha bibliográfica).
\imprimirfolhaderosto*


% Ficha catalográfica.
% (*) Escolher entre as versões de ficha catalográfica abaixo (comente aquela que não quiser usar).

% Versão 1: caso a biblioteca da sua universidade lhe forneça um PDF (adequar o nome do arquivo).
% \begin{fichacatalografica}
%     \includepdf{include-fichacatalografica.pdf}
% \end{fichacatalografica}

% Versão 2: caso você tenha que inserir sua própria ficha catalográfica.
% (*) Neste caso, preencher palavras-chave e adicione co-orientador (se houver).



% Folha de aprovação.
% (*) Escolher entre as versões de ficha catalográfica abaixo (comente aquela que não quiser usar).

% Versão 1: cópia digitalizada da folha de aprovação assinada pela banca.
% \includepdf{include-folhadeaprovacao.pdf}

% Versão 2: folha de aprovação em branco.
% (*) Ajustar a data e os nomes dos participantes da banca.
\begin{folhadeaprovacao}
  \begin{center}
    {\ABNTEXchapterfont\large\imprimirautor}
    \vspace*{\fill}\vspace*{\fill}
    \begin{center}
      \ABNTEXchapterfont\bfseries\Large\imprimirtitulo
    \end{center}
    \vspace*{\fill}
    \hspace{.45\textwidth}
    \begin{minipage}{.5\textwidth}
        \imprimirpreambulo
    \end{minipage}%
    \vspace*{\fill}
   \end{center}
   Trabalho aprovado. \imprimirlocal, 03 de agosto de 2017:
   \assinatura{\textbf{\imprimirorientador} \\ Orientador} 
   \assinatura{\textbf{Prof. Dra. Maria Cristina Rangel} \\ Convidado 1}
   \assinatura{\textbf{Prof. Me. Edmar Hell Kampke} \\ Convidado 2}
   %\assinatura{\textbf{Professor} \\ Convidado 3}
   %\assinatura{\textbf{Professor} \\ Convidado 4}
   \begin{center}
    \vspace*{0.5cm}
    {\large\imprimirlocal}
    \par
    {\large\imprimirdata}
    \vspace*{1cm}
  \end{center}  
\end{folhadeaprovacao}


% Dedicatória.
% (*) Escrever dedicatória ou remover/comentar seção.



% Agradecimentos.
% (*) Escrever agradecimentos ou remover/comentar seção.


% Epígrafe.
% (*) Escrever epígrafe ou remover/comentar seção.



% Resumo em português.
% (*) Escrever resumo e palavras-chave.
\setlength{\absparsep}{18pt}
\begin{resumo}
Este trabalho propõe um estudo de algoritmos de menor caminho em grafos, tanto algoritmos de grafos estáticos quanto dinâmicos, descrevendo sua forma, estudando seu funcionamento, testando seu desempenho computacional e analisando em quais situações melhor se aplicam. Inicialmente é descrito o algoritmo clássico de Dijkstra, no qual é apresentado seu funcionamento, seguido de suas versões de implementação baseados em estrutura de dados que representam a fila de prioridade de ordenamento dos vértices. Em seguida é apresentado o algoritmo busca A*, que pode ser visto como uma adaptação do algoritmo de Dijkstra. É apresentado a estratégia do uso de heurística, tanto a admissível quanto não-admissível, a qual o algoritmo busca A* utiliza, e o impacto causado por ela. Em seguida, é apresentado o conceito de grafo dinâmico e os algoritmos ARA* e AD*, sendo que o AD* é uma adaptação do ARA*. São descritos seus funcionamentos e a utilização da estratégia da heurística inflada. Finalmente então, é descrito os testes computacionais realizados para cada algoritmo, a descrição das instâncias escolhidas, análise dos resultados obtidos e as conclusões sobre em quais situações o uso desses algoritmos melhor se aplicam.

\textbf{Palavras-chaves}: grafos estáticos, grafos dinâmicos, algoritmos de menor caminho, desempenho, busca A*, Dijkstra, ARA*, AD*.
\end{resumo}

% Insere lista de ilustrações.
\pdfbookmark[0]{\listfigurename}{lof}
\listoffigures*
\cleardoublepage

% Insere lista de tabelas.
\pdfbookmark[0]{\listtablename}{lot}
\listoftables*
\cleardoublepage

% Insere o sumário.
\pdfbookmark[0]{\contentsname}{toc}
\tableofcontents*
\cleardoublepage



%%% Início da parte de conteúdo do documento. %%%

% Marca o início dos elementos textuais.
\textual

% Inclusão dos capítulos.
% (*) Para facilitar a organização, os capítulos foram divididos em arquivo separados e colocados dentro da.
% pasta capitulos/. Caso o aluno prefira trabalhar com um só arquivo, basta substituir os comandos \include 
% pelos conteúdos dos arquivos que estão sendo incluídos, excluindo a pasta capitulos/ em seguida.
% ==============================================================================
% TCC - Nome do Aluno
% Capítulo 1 - Introdução
% ==============================================================================
\chapter{Introdução}
\label{sec-intro}
Na computação muitas aplicações necessitam considerar um conjunto de conexões entre dados e uso de algoritmos sobre esses dados para poder responder perguntas, como exemplo, se existe um caminho entre dois dados distintos seguindo por essas conexões, qual a menor distância entre eles ou ainda quantos dados podemos alcançar a partir de um outro determinado dado. Para modelar tais situações, utilizamos um tipo abstrato chamado grafo \cite{ziviani2004projeto}. Um grafo G = (V,E) consiste de um conjunto não vazio V de vértices, que representam os dados, e um conjunto que pode ser vazio de arestas E, que representam a ligação entre esses dados \cite{diestel}. A figura \ref{fig-intro-exemplografo} mostra um exemplo de um grafo. Exemplos de modelagens por grafos são a ligação entre cidades em que as arestas representam as distâncias entre elas e a representação de uma rede interna de computadores. 

\begin{figure}[H]
\centering
\includegraphics[width=.65\textwidth]{figuras/grafo-exemplo} 
\caption{Exemplo de um grafo.}
\label{fig-intro-exemplografo}
\end{figure}

Um problema clássico da literatura envolvendo grafos é o cálculo de caminho mínimo \cite{moura2010estudo}. Nele desejamos obter o menor caminho entre dois pontos específicos do grafo representado. A sua modelagem é feita representando as arestas com determinados pesos que podem significar o tempo decorrente entre executar tarefas, o custo de transmitir informações entre locais, quantidades específicas a serem transportadas entre um local e outro e etc \cite{drozdek2012data}. 

Dentre as modelagens realizadas para resolver o problema do caminho mínimo, uma muito utilizada é a definição do menor caminho entre dois pontos geográficos tendo como aplicação prática o uso de softwares do tipo GPS. Nela representamos as interseções das ruas, caminhos ou rodovias como o vértices do grafo e as distâncias entre essas interseções (as próprias ruas, caminhos ou rodovias) como as arestas e tendo seus respectivos pesos como sendo a distâncias entre essas interseções.

%A figura X mostra o exemplo desta modelagem abordada [a ser colocado ainda professora].

A motivação deste trabalho é o estudo dos algoritmos de menor caminho em grafos, desde os clássicos como o algoritmo de Dijkstra \cite{dijkstra1959note} até algoritmos mais recentes propostos como o \textit{Anytime Dynamic} A* (AD*) \cite{likhachev2008anytime}. Tem por  objetivo verificar o desempenho e a eficácia destes algoritmos, averiguar o impacto que as estruturas de dados utilizadas para resolver o problema causam e analisar quais situações os algoritmos estudados melhor se aplicam.

\section{Revisão bibliográfica e trabalhos correlatos}
\label{sec-intro-correlatos}
O algoritmo de Dijkstra é amplamente difundido na literatura, sendo que é utilizado como base para o estudo desse algoritmo os livros de \citeonline{cormen2009introduction} e \citeonline{drozdek2012data}, em que abordam a descrição do algoritmo, estruturas de dados e a análise computacional das mesmas.

O algoritmo A* é bastante difundido na literatura. Foi proposto por \citeonline{hart1968formal}. Ele é descrito no livro de \citeonline{russell1995modern} e no artigo de \citeonline{likhachev2008anytime}. O pseudo-código descrito neste livro e artigo são usados como base para a implementação desse algoritmo neste projeto.

O estudo dos algoritmos ARA* e AD* são variações do algorimto A* para a solução do problema de caminho mínimo em grafos dinâmicos e foram propostos por \citeonline{likhachev2008anytime}. O artigo \cite{moura2010estudo} investiga o comportamento desses algoritmos aplicados a mapas reais. 

O trabalho de \citeonline{larkin2014back} analisa o uso de diversas estruturas de dados como fila de prioridades e seus respectivos impactos, ressaltando diferenças entre a proposta teórica dessas estruturas e a aplicação prática.
\section{Metodologia de pesquisa}
\label{sec-intro-metodologia}
Neste trabalho estudamos os algoritmos de Dijkstra e A* para a solução do problema de caminho mínimo em grafos estáticos, e os algoritmos ARA* e AD* para grafos dinâmicos.

Todos os algoritmos estudados foram implementados na linguagem Java, a versão do compilador utilizado é Java 1.8.0\_131. Os testes a serem realizados e as instâncias utilizadas estão descritos no capítulo \ref{sec-testes}.
\newpage
\section{Organização dos capítulos}
\label{sec-intro-organizaocao}
Nos próximos capítulos este trabalho está estruturado da seguinte forma:
\begin{description}
%\item[Capítulo \ref{sec-intro}:] Esta introdução.
\item[Capítulo \ref{sec-algoritmosestaticos}:] Apresenta a descrição do algoritmo clássico de Dijkstra, as versões implementadas baseadas em estrutura de dados diferentes e suas respectivas descrições. Apresenta também a descrição do algoritmo busca A*, sua diferença em relação ao Dijkstra e uma descrição do uso da estratégia de heurísticas e como elas são classificadas;
\item[Capítulo \ref{sec-dinamicos}:] Apresenta a descrição dos algoritmos ARA* e AD*, a estratégia de heurística inflada e uma breve descrição de grafos dinâmicos, nos quais o algoritmo AD* se aplica;
\item[Capítulo \ref{sec-testes}:] Apresenta a descrição dos testes realizados, seus respectivos resultados e uma análise desses para cada algoritmo abordado;
\item[Capítulo \ref{sec-conclusao}:] Traz a conclusão geral deste trabalho e possíveis trabalhos futuros.
\end{description}
% ==============================================================================
% TCC - Nome do Aluno
% Capítulo 2 - Referencial Teórico
% ==============================================================================
\chapter{Algoritmo de Dijkstra}
\label{sec-dijkstra}

\section{O Algoritmo}
\label{sec-dijkstra-algoritmo}
O algoritmo de Dijkstra foi proposto por Edgar W. Dijkstra em 1959 \cite{dijkstra1959note}. Ele tem por objetivo definir o menor caminho partindo do vértice origem $v_{s}$ e chegando a todos os demais vértices $v_{i}$ do grafo $G = (V,E)$. Para garantir a viabilidade do algoritmo, assume-se que todos os pesos $w( u, v )$ sejam maiores ou iguais a zero para toda aresta $E$ do grafo $G$ \cite{cormen2009introduction}.

A seguir é apresentado o pseudocódigo do algoritmo conforme descrito em \citeonline{drozdek2012data}.
%\begin{verbatim}
%CÓDIGO AQUI
%\end{verbatim}

\begin{lstlisting}[ mathescape, label=lst-dijkstra-codigo, caption=Algoritmo de Dijkstra., float=htpb]
DijkstraAlgorithm(weighted simple digraph, vertex first)
	for all vertices v
		currDist(v) = $\infty$;
	currDist(first) = 0;
	toBeChecked = all vertices;
	while toBeChecked is not empty
		v = a vertex in toBeChecked with minimal currDist(v);
		remove v from toBeChecked;
		for all vertices u adjacent to v and in toBeChecked
			if currDist( u ) > currDist( v ) + weight( edge(vu) )
				currDist( u ) = currDist( v ) + weight( edge(vu) );
				predecessor( u ) = v;
\end{lstlisting}

%\begin{verbatim}
%DijkstraAlgorithm(weighted simple digraph, vertex first)
%	for all vertices v
%	currDist(v) = infinite;
%	currDist(first) = 0;
%	toBeChecked = all vertices;
%	while toBeChecked is not empty
%		v = a vertex in toBeChecked with minimal currDist(v);
%		remove v from toBeChecked;
%		for all vertices u adjacent to v and in toBeChecked
%		if currDist( u ) > currDist( v ) + weight( edge(vu) )
%			currDist( u ) = currDist( v ) + weight( edge(vu) );
%			predecessor( u ) = v;
%\end{verbatim}

O algoritmo inicia atribuindo o valor inicial de cada distância de cada vértice do grafo igual a $\infty$, com exceção do vértice inicial $v_{s}$ que será iniciado por 0. Em seguida todos os vértices são adicionados ao conjunto dos "toBeChecked" ("aSeremChecados"). Feito isso, inicia-se o processo iterativo: seleciona-se o vértice $v$ de menor custo que esteja dentro do conjunto "toBeChecked", retira-se ele do conjunto e a partir dele, para cada vértice adjacente $u$ de $v$, verifica-se se a distância atual calculada de $u$ é maior do que a distância calculada de $v$ mais o valor referente ao peso da aresta de $v$ e $u$ (origem em $v$). Caso seja verdade, a distância atual de $u$ é substituída pela soma da distância atual de $v$ mais o peso da aresta de $v$ e $u$ (este valor corresponde à distância do vértice de origem $v_{s}$ até $u$), além de definir o antecessor $u$ como $v$. Repete-se o passo iterativo até que o conjunto "toBeChecked" esteja vazio\footnote{Em linguagens de programação, é costume substituir o valor $\infty$ pelo maior número representativo do tipo da variável selecionado para representar a distância. Por exemplo na linguagem C, caso se utilize o valor int (inteiro) para representar a distância, a atribuição inicial será dado pela constante INT\underline{\space}MAX  definida pela biblioteca "limits.h", que representa o maior valor numérico representado por esse tipo de variável.}.
\newpage
Ao final do algoritmo, teremos o conjunto de predecessores de cada vértice do grafo, e a partir deste, poderemos definir a rota para qualquer vértice do grafo partindo de $v_{s}$. O caminho retornado é garantidamente ótimo como mostra \citeonline{cormen2009introduction}.

A figura \ref{fig-dijkstra-algoritmo-grafo} mostra um exemplo de aplicação do algoritmo a um grafo.

\begin{figure}[H]
\centering
\includegraphics[width=1.\textwidth]{figuras/grafo-dijkstra} 
\caption{Aplicação do algoritmo de Dijkstra tendo o vértice "A" como origem. As arestas pintadas de preto correspondem a rota calculada a todos os demais vértices.}
\label{fig-dijkstra-algoritmo-grafo}
\end{figure}

Inicialmente a distância do vértice "A" é atribuído como zero enquanto de todos os demais é atribuído como infinito $\infty$. Inicia-se o processo iterativo a partir de "A" que explora os seus vértices adjacentes, que neste caso só tem um que é o vértice "C". A distância de "C" é calculada como o valor da distância de "A" mais o peso da aresta "AC" (que é igual a 1), e o vértice que é atribuído como antecessor de "C" é "A". Em seguida é escolhido o vértice cuja a distância seja a menor dentro do conjunto "toBeChecked" que neste caso é o próprio "C". Do vértice "C" explora-se os vértices adjacentes dele "D" e "B". Suas respectivas distâncias são atribuídas como a distância de "C" mais o peso de suas respectivas arestas (currDist( B ) = 7 + 1; currDist( D ) = 3 + 1), além de atribuir como "C" os seus respectivos vértices antecessores. Disso novamente, escolhe-se o vértice com menor distância em "toBeChecked" que neste caso será o D (currDist( D ) = 4 < currDist( C ) = 8). De "D" é explorado seu vértice adjacente "E" que é atribuído seu valor currDist( E ) como sendo 5 (3 + 1 + 1) e seu vértice antecessor como "D". Busca-se novamente o menor valor dos vértices em "toBeChecked" que neste caso será o "E" (currDist( E ) = 5 < currDist( B ) = 8). Dele explora-se os seus vértices adjacentes "B" e "A". O único valor da distância que é alterado é o de "B" pois o caminho vindo por "E" (A->C->D->E = 1 + 3 + 1 + 1) é menor do que o vindo por "C"  (A->C->B = 1 + 7). Finalmente, o vértice "B" é explorado, mas nenhum de seus vértices adjacentes tem o valor de sua distância alterado pois o menor caminho para eles já foi encontrado.

\section{Versões do Algoritmo implementadas e suas Estrutura de Dados}
\label{sec-dijkstra-versoes}
Para este projeto de graduação será implementada três versões do algoritmo de Dijkstra baseados em estruturas de dados diversas que implicam em tempos computacionais diferentes \cite{cormen2009introduction}.

As versões implementadas são o Dijkstra Canônico (descrito a seguir), Dijkstra Heap Binário (subseção \ref{sec-dijkstra-versoes-heap}) e Dijkstra Heap de Fibonacci (subseção \ref{sec-dijkstra-versoes-fibonacci}), todas baseadas em \citeonline{cormen2009introduction, drozdek2012data}.

Para a versão Dijkstra Canônico o algoritmo utiliza de um vetor para armazenar as distâncias calculadas pelo algoritmo (o indíce dos vértices correspondem ao indíce do vetor em que são armazenados), e a cada passo iterativo (conforme demonstrado pelo algoritmo na seção \ref{sec-dijkstra-algoritmo}), uma busca linear é realizada para determinar o vértice (fora do conjunto "toBeChecked") cuja a distância é menor dentre todas as outras. O tempo computacional para esse caso é $O(|V^{2}|)$ \cite{drozdek2012data}.

\subsection{Dijkstra Heap Binário}
\label{sec-dijkstra-versoes-heap}
Para esta implementação, será utilizada a estrutura de dados heap binária mínima como fila de prioridade. Heaps binária podem ser descritas como árvores binárias que possuem as seguintes propriedades \cite{drozdek2012data}:
\begin{enumerate}
 \item O valor de cada nodo não é maior do que os valores guardados em cada um de seus filhos.
 \item A árvore é perfeitamente balanceada, e as folhas no último nível estão todas mais a esquerda.
\end{enumerate}

Um exemplo de estrutura Heap Binário representada tanto como árvore como vetor pode ser visualizado nas figuras \ref{fig-dijkstra-heapbinario} e \ref{fig-dijkstra-heapvetor} respectivamente.

\begin{figure}[H]
\centering
\includegraphics[width=.95\textwidth]{figuras/Heap} 
\caption{Exemplo de Heap Binário representado como árvore.}
\label{fig-dijkstra-heapbinario}
\end{figure}

\begin{figure}[H]
\centering
\includegraphics[width=.60\textwidth]{figuras/Heap-vetor}
\caption{Representação do Heap Binário da figura \ref{fig-dijkstra-heapbinario} como vetor.}
\label{fig-dijkstra-heapvetor}
\end{figure}

A disposição dos elementos da árvore no vetor segue as seguintes relações entre nós pai, filho-direita e filho-esquerda:
%\begin{equation}
%Pai(i): \lfloor i/2 \rfloor
%\end{equation}
%\begin{equation}
%Filho-esquerda(i): 2*i
%\end{equation}
%\begin{equation}
%Filho-direita(i): 2*i+1
%\end{equation}
\begin{description}
\item[Pai($i$):] $\lfloor i/2 \rfloor$
\item[Filho-esquerda($i$):] $2*i$
\item[Filho-direita($i$):] $2*i+1$
\end{description}
Onde $i \in \mathbb{N}$ e $i \subset [1, n]$, sendo que $i$ representa o índice do elemento no vetor e $n$ o número de elementos da árvore.

 Para efeito de exemplo (observe as figuras \ref{fig-dijkstra-heapbinario} e \ref{fig-dijkstra-heapvetor} para constatação), o nó que está contido na posição 4 do vetor possui como pai o nó de posição 2 ($\lfloor 4 / 2 \rfloor = 2$), tem como filho da esquerda o nó de posição 8 ($2*4 = 8$) e filho da direita o nó de posição 9 ($2*4+1 = 9$).

%\begin{equation}
%\forall i, i \in \mathbb{N} [1,n]
%\end{equation}

A vantagem de se usar essa estrutura de dados reside no fato de suas operações de inserção, extração de mínimo e reconstrução da heap possuirem tempo computacional de $O(\lg n)$. Por consequência, o tempo computacional para este caso é de $O(|E| \lg |V|)$ \cite{cormen2009introduction}.

\subsection{Dijkstra Heap de Fibonacci}
\label{sec-dijkstra-versoes-fibonacci}
A Heap de Fibonacci consiste de uma coleção de árvores que seguem a regra de árvore heap mínima, ou seja, os nós pais são maiores ou iguais aos nós filhos. Os nós raízes de cada árvore são interligados por uma lista circular duplamente encadeada. Um ponteiro chamado "raiz mínima" aponta para o nó de menor valor.

Sua característica é que operações de adição são executadas de uma maneira "preguiçosa", não procurando criar uma forma para as árvores (como por exemplo, deixa-la balanceada), apenas as adicionados-as a lista principal de raízes. Por consequência, operações de inserção possuem tempo computacional $O(1)$ \cite{cormen2009introduction}.

\begin{figure}[H]
\centering
\includegraphics[width=.54\textwidth]{figuras/fibonacci-heap1} 
\caption{Exemplo de Heap de Fibonacci (os números no canto superior esquerdo de cada nodo correspondem ao grau de cada um, ou seja, o número de filhos).}
\label{fig-dijkstra-heapfibonacci1}
\end{figure}

Para operações de extração de mínimo o tempo computacional é mais custoso. Isso é devido ao fato de que quando o mínimo é retirado, a heap precisa ser reorganizada de forma que sua propriedade principal não seja violada e o um novo mínimo seja determinado. Para isso a operação de extração de mínimo se dá em três etapas. Primeiro é retirado o mínimo da heap (se caso o mínimo possua nós filhos, eles são colocados na lista principal de raízes) e o seu vizinho é assimilado como o novo mínimo provisório. Agora precisamos definir quem é o novo mínimo e para isso teremos que verificar todos os demais nós raízes. Com o intuito de diminuir o número de nós raízes é que o segundo passo é aplicado.  Ele consiste em lincar raízes com o mesmo número de grau (grau corresponde ao número de filhos que cada nodo possui) e para cada par de nodos licandos, verifica-se qual dos dois é menor. O que for o menor será o nodo pai e outro por consequência será o nodo filho. Após a lincagem de todos os nodos com mesmo número de grau, uma busca linear é realizada para se determinar o menor elemento entre os nodos raízes restantes\footnote{Para otimizar a busca de nodos com o mesmo número de grau é utilizado um vetor auxiliar de tamanho mínimo ao maior grau de um nodo da estrutura. Esse vetor contém ponteiros para os nodos e a posição desse nodo no ponteiro corresponde ao grau do nodo. Por exemplo, se um nodo possui grau 3, ele ocupará a posição de número 3 no vetor. Quando um nodo da lista é referenciado na posição que já está ocupado, o processo de lincagem é feito conforme descrito.}. O tempo computacional para a extração de mínimo é $O(\lg n)$ \cite{cormen2009introduction}.

%\footnote{Para otimizar a busca de nodos com o mesmo número de grau é utilizado um vetor auxiliar de tamanho mínimo ao maior grau de um nodo da estrutura. Esse vetor contém ponteiros para os nodos e a posição desse nodo no ponteiro corresponde ao grau do nodo. Por exemplo, se um nodo possui grau 3, ele ocupará a posição de número 3 no vetor. Quando um nodo da lista é referenciado na posição que já está ocupado, o processo de lincagem é feito conforme descrito.}

\begin{figure}[H]
\centering
\includegraphics[width=.54\textwidth]{figuras/fibonacci-heap2} 
\caption{Heap de Fibonacci da figura \ref{fig-dijkstra-heapfibonacci1} após a operação de extração de mínimo.}
\label{fig-dijkstra-heapfibonacci2}
\end{figure}


Finalmente para a operação de mudança de chave de um determinado nodo, e após a mudança realizada em tempo constante ($O(1)$), é verificado se a propriedade heap foi violada. Se caso sim, esse nodo é cortado de seu nodo pai e colocado junto a lista principal. Se o pai não pertencer a lista de nodos raízes, ele é "marcado" ("pintado") e caso já estivesse "marcado" ele também é cortado e seu pai é "marcado". Esse processo continua subindo até encontrarmos um nodo pai "não marcado" ou um nodo raiz. Após esse processo recursivo ter terminado, verifica-se se nodo modificado inicialmente é menor do que o nodo mínimo atual. Se caso sim, o novo nodo mínimo é o modificado. O tempo computacional é $O(1)$ \cite{cormen2009introduction}.

Por consequência, o tempo computacional aplicado para o algoritmo de Dijkstra é de $O(|V|\lg |V| + |E|)$ \cite{cormen2009introduction}.
% ==============================================================================
% TCC - Nome do Aluno
% Capítulo 3 - Especificação de Requisitos
% ==============================================================================
\chapter{Algoritmo A*}
\label{sec-aestrela}

\section{O Algoritmo}
\label{sec-aestrela-algoritmo}
O algoritmo A* (lê-se "A estrela") também conhecido como busca A* é um algoritmo de busca informatizada em grafos. Foi proposta originalmente em (buscar referência) e pode ser visto como uma adaptação do algoritmo de Dijkstra (apresentado no capítulo \ref{sec-dijkstra}) em que, ao invés de se calcular a melhor rota de um ponto de partida para todos os demais vértices do grafo, se estabelece uma boa rota (ou mesmo a rota ótima\footnote{A garantia do valor ótimo do algoritmo depende de fatores que serão discutidos na subseção \ref{sec-aestrela-algoritmo-heuristica}.}) partindo do vértice origem a um vértice destino. Isso é feito realizando "podas" do caminho de forma que não seja necessário visitar todos os vértices, apenas os mais promissores.

A seguir é apresentado o algoritmo A* adaptado de \citeonline{likhachev2008anytime} sobre o algoritmo de Dijkstra apresentado na seção \ref{sec-dijkstra-algoritmo}. 

\begin{lstlisting}[ mathescape, label=lst-aestrela-codigo, caption=Algoritmo A*, float=htpb]
$A^{*}$Algorithm(weighted simple digraph, vertex first, vertex goal)
	for all vertices v
		g(v) = $\infty$;
	g(first) = 0;
	toBeChecked = all vertices;
	while goal is in toBeChecked
		v = a vertex in toBeChecked with minimal f(v);
		remove v from toBeChecked;
		for all vertices u adjacent to v and in toBeChecked
			if g( u ) > g( v ) + weight( edge(vu) )
				g( u ) = g( v ) + weight( edge(vu) );
				predecessor( u ) = v;
				update u in toBeChecked with f(u) = g(u) + h(u);
\end{lstlisting}

O algoritmo segue em sua essência como um Dijkstra adaptado. Iniciamos a distância de todos os vértices g(v) como sendo $\infty$\footnote{Vide nota de rodapé da seção \ref{sec-dijkstra-algoritmo}.} (o valor de g(v) corresponde ao valor da distância calculada do vértice origem "first" até o vértice "v") com exceção do vértice origem, cujo valor atribuído é zero. Adicionamos todos os vértices ao grupo dos "toBeChecked" \footnote{Algumas literaturas designam esse conjunto como OPEN.}. Feito isso inicia-se o processo iterativo: enquanto o vértice "goal" estiver dentro do conjunto do "toBeChecked" (ou seja, o vértice "goal" não foi alcançado ainda pelo algoritmo), o vértice com menor valor f(v) é retirado do conjunto "toBeChecked" e para cada vértice adjacente u de v, verifica-se se o valor de g( u ) atual é maior que g( v ) mais o peso da aresta entre v e u (edge(vu)). Se caso for verdade, o valor de g( u ) é atualizado para g( v ) mais o peso da aresta entre v e u, e v é marcado como o predecessor de u. O valor do peso do vértice u é atualizado na fila de prioridades utilizada (como a heap binária, descrita na subseção \ref{sec-dijkstra-versoes-heap}) com o valor f( u ) = g( u ) + h( u ).

Observe que para o algoritmo A*, diferente do que ocorre em Dijkstra, não utiliza o valor de g( u ) (valor da distância calculada do vértice origem "first" até o vértice "u") como chave de ordenamento da fila de prioridade, mas sim esse valor acrescido de h( u ). Observe também que o algoritmo para ao ser removido o vértice destino ("goal") da lista do "toBeChecked" em contrapartida ao Dijkstra que calcula para todos os vértices do grafo.

O termo h( u ) significa o valor heurístico que corresponde a uma estimativa da distância de u ao vértice destino "goal". É devido a esse valor que o algoritmo A* realiza "podas" no número de vértices a serem checados, buscando os mais promissores, já que o valor heurístico faz com que os vértices cujas estimativas sejam mais próximas do vértice destino ("goal") sejam colocados mais a frente na fila de prioridades e por consequência, sejam calculadas primeiros. E assim é mais provável que o vértice destino seja alcançado antes e tenha sua rota calculada, terminando o algoritmo. O valor h( u ) é classificado como admissível e não-admissível cujo significado será dado na subseção \ref{sec-aestrela-algoritmo-heuristica}.

A figura \ref{fig-aestrela-algoritmo-mapa1} contida em \citeonline{russell1995modern} mostra um exemplo de aplicação do algoritmo.

\begin{figure}[H]
\centering
\includegraphics[width=.90\textwidth]{figuras/Aestrela-mapa1} 
\caption{Mapa da Romênia com os valores das distâncias entre as cidades, e a distância euclidiana até Bucareste.}
\label{fig-aestrela-algoritmo-mapa1}
\end{figure}

Um viajante deseja partir da cidade de Arad com destino a Bucareste buscando percorrer o menor caminho entre essas duas cidades. Para isso é utilizado o algoritmo A* que explora o grafos conforme descrito na figura \ref{fig-aestrela-algoritmo-mapa2}, tendo como heurística utilizada, a distância euclidiana entre todas as cidades e Bucareste (distâncias também contida na figura \ref{fig-aestrela-algoritmo-mapa1}).

\begin{figure}[H]
\centering
\includegraphics[width=.90\textwidth]{figuras/Aestrela-mapa2} 
\caption{Desenvolvimento do algoritmo A*.}
\label{fig-aestrela-algoritmo-mapa2}
\end{figure}




%\citeonline{russell1995modern,cormen2009introduction}.

%\begin{lstlisting}[ mathescape, label=lst-aestrela-codigo, caption=Algoritmo A*, float=htpb]
%Entrada: Grafo G, vértice inicial $v_{i}$, vértice destino $v_{f}$
%	para todo vértice $s \in V(G), s \neq vi$ faça
%		$g(s) = \infty$
%	fim para
%	$g(v_{i}) = 0$
%	para todo vértice $s \in V(G)$ faça
%		anterior($s$) = -1
%	fimpara
%	OPEN = {$v_{i}$}
%	enquanto $v_{f}$ não é expandido faça
%		s = desenfila(OPEN)
%		para cada sucessor $s'$ de $s$ faça
%			se $g(s') > g(s) + c(s,s')$ então
%				$g(s') = g(s) + c(s,s')$
%				anterior($s'$) = s
%				insira/atualize $s'$ em OPEN pelo valor de f
%			fim se
%		fim para
%	fim enquanto
%retorna anterior
%\end{lstlisting}



\subsection{Heurísticas admissíveis e não-admissíveis}
\label{sec-aestrela-algoritmo-heuristica}  
O fator heurístico h( u ) é uma estimativa da distância entre o vértice "u" e o vértice "goal". Ela é chamada de \textbf{admissível} quando o valor da estimativa garantidamente não superestima o valor da distância real entre "u" e "goal" \cite{russell1995modern}. Um exemplo clássico usado de heurística admissível é a distância euclidiana, já que a menor distância entre dois pontos é uma reta \cite{russell1995modern}.

O cálculo da distância euclidiana porém, nem sempre é a forma mais rápida em termos computacionais, já que geralmente ela é calculada em termos dos pontos geográficos do vértice e esse cálculo envolve exponenciação e radiação. É por isso que existe o uso de heurísticas \textbf{não-admissíveis}, que são estimativas que visam a usar cálculos mais simples, porém não há a garantia que essa distância superestime a distância real entre "u" e "goal", e com isso não há garantia que o melhor caminho seja encontrado.

Exemplos de heurísticas não-admissíveis:
\begin{itemize}
\item Distância Manhattan: h( u ) = $| x_{u} - x_{goal} | + | y_{u} - y_{goal}|$;
\item Atalho Diagonal: h( u ) = $\sqrt{2} * | y_{u} - y_{goal}| + ( | x_{u} - x_{goal} | - | y_{u} - y_{goal}| )$ [Se a distância $| x_{u} - x_{goal} | > | y_{u} - y_{goal}|$] || $\sqrt{2} * | x_{u} - x_{goal}| + ( | y_{u} - y_{goal}| - | x_{u} - x_{goal} | )$ [Se a distância $| x_{u} - x_{goal} | < | y_{u} - y_{goal}|$];
\end{itemize}

\section{Experimentos Computacionais}
\label{sec-aestrela-experimentos}

Para os experimentos computacionais serão utilizados as mesmas instâncias descritas em subseção \ref{sec-dijkstra-experimentos}. Serão comparados quatro versões de algoritmos: o algoritmo de Dijkstra descrito no capítulo \ref{sec-dijkstra}, o algoritmo de Dijkstra adaptado onde o algoritmo é parado assim que se é explorado o vértice objetivo, o algoritmo A* onde se utiliza a heurística admissível distância euclidiana e o algoritmo A* onde se utiliza a heurística não-admissível distância Manhattan, todas sumarizadas na tabela X.

% ==============================================================================
% TCC - Nome do Aluno
% Capítulo 3 - Projeto Arquitetural e Implementação
% ==============================================================================
\chapter{Algoritmos Dinâmicos}
\label{sec-dinamicos}

Texto.

% ==============================================================================
% TCC - Nome do Aluno
% Capítulo 5 - Testes Computacionais
% ==============================================================================
\chapter{Testes Computacionais}
\label{sec-testes}

\section{Algoritmo de Dijkstra}
\label{sec-dijkstra-experimentos}
Para a realização dos experimentos computacionais será rodado instâncias de grafos que representam malhas rodoviárias reais. Todas elas descritas nas tabelas \ref{tbl-dijkstra-instancias} e \ref{tbl-dijkstra-configuracao}, e disponíveis no sítio eletrônico \url{http://www.dis.uniroma1.it/challenge9/download.shtml} (acesso em 28 de janeiro de 2017).
\begin{table}[H]
\caption{Instâncias a serem rodadas pelo algoritmo de Dijkstra em suas três versões.}
\label{tbl-dijkstra-instancias}
\centering
\begin{adjustbox}{max width=\textwidth}
\begin{tabular}{|c|c|}
\hline 
\textbf{Nome Instância} & \textbf{Descrição} \\ 
\hline 
USA-road-d.NY.gr & Representa a malha viária do estado de Nova Iorque, Estados Unidos \\ 
\hline 
USA-road-d.BAY.gr & Representa a malha viária da bahia de São Francisco, Califórnia, Estados Unidos \\ 
\hline 
USA-road-d.COL.gr & Representa a malha viária do estado do Colorado, Estados Unidos \\ 
\hline 
USA-road-d.FLA.gr & Representa a malha viária do estado da Flórida, Estados Unidos \\ 
\hline 
\end{tabular}
\end{adjustbox}
\end{table}

\begin{table}[H]
\caption{Configuração dos grafos correspondestes as malhas viárias descritas na tabela \ref{tbl-dijkstra-instancias}.}
\label{tbl-dijkstra-configuracao}
\centering
\begin{adjustbox}{max width=\textwidth}
\begin{tabular}{|c|c|c|}
\hline 
\textbf{Nome Instância} & \textbf{Número de Vértices |V|} & \textbf{Número de Arestas |E|} \\ 
\hline 
USA-road-d.NY.gr
 & 264.346
 & 733.846
 \\ 
\hline 
USA-road-d.BAY.gr
 & 321.270
 & 800.172
 \\ 
\hline 
USA-road-d.COL.gr
 & 435.666
 & 1.057.066
 \\ 
\hline 
USA-road-d.FLA.gr
 & 1.070.376
 & 2.712.798
 \\ 
\hline 
\end{tabular}
\end{adjustbox} 
\end{table}

\subsection{Resultados obtidos}
\label{sec-dijkstra-experimentos-resultados}
Os resultados dos testes obtidos estão descritos a seguir.

\begin{figure}[H]
\centering
\includegraphics[width=.90\textwidth]{figuras/dijkstra-tempos} 
\caption{Tempos computacionais obtidos pelos Métodos de Dijkstra empregados (tempo em escala logarítmica).}
\label{fig-dijkstra-resultados-tempos}
\end{figure}

\begin{figure}[H]
\centering
\includegraphics[width=.90\textwidth]{figuras/speed-up-dijkstra} 
\caption{Ganho relativo dos Métodos do Algoritmo de Dijkstra sobre o Canônico.}
\label{fig-dijkstra-resultados-speedup}
\end{figure}


\begin{table}[H]
\caption{Tempos Computacionais obtidos pelo algoritmo de Dijkstra em suas diferentes versões (tempo em milissegundos).}
\label{tbl-dijkstra-resultados-tempos}
\centering
\begin{adjustbox}{max width=\textwidth}
\begin{tabular}{|c|c|c|c|}
\hline
\textbf{Nome Instância} & \textbf{Tempo Canônico} & \textbf{Tempo Heap Binário} & \textbf{Tempo Heap Fibonacci} \\ \hline
USA-road-d.NY.gr        & 65287                   & 153                         & 302                           \\ \hline
USA-road-d.BAY.gr       & 256690                  & 248                         & 755                           \\ \hline
USA-road-d.COL.gr       & 181687                  & 318                         & 627                           \\ \hline
USA-road-d.FLA.gr       & 2909655                 & 6601                        & 8732                          \\ \hline
\end{tabular}
\end{adjustbox}
\end{table}


\begin{table}[H]
\caption{Ganho relativo sobre o Dijkstra Canônico.}
\label{tbl-dijkstra-resultados-speedup}
\centering
\begin{adjustbox}{max width=\textwidth}
\begin{tabular}{|c|c|c|}
\hline
\textbf{Nome Instância} & \textbf{Ganho Heap Binário} & \textbf{Ganho Heap Fibonacci} \\ \hline
USA-road-d.NY.gr        & 42.671\%                       & 21.618\%                         \\ \hline
USA-road-d.BAY.gr       & 103.504\%                      & 33.999\%                         \\ \hline
USA-road-d.COL.gr       & 57.134\%                       & 28.977\%                         \\ \hline
USA-road-d.FLA.gr       & 44.079\%                       & 33.322\%                         \\ \hline
\end{tabular}
\end{adjustbox}
\end{table}

\subsection{Análise dos resultados}
\label{sec-dijkstra-experimentos-analise}
Podemos observar que o Dijkstra Canônico elevou um tempo consideravelmente grande (a instância USA-road-d.FLA.gr por exemplo levou aproximadamente 48 minutos). Já o uso de estrutura de dados impactou consideravelmente no ganho do tempo sendo que o Heap Binário teve um ganho médio de 61.847\% com relação ao Dijkstra Canônico enquanto Dijkstra Heap de Fibonacci teve um ganho médio de 29.479\% (vide tabela \ref{tbl-dijkstra-resultados-speedup}).

Com relação ao resultado obtido pelos métodos do Heap Binário e Heap de Fibonacci, ele é de certo modo inesperado, já que conforme demonstrados nas subseções \ref{sec-dijkstra-versoes-heap} e \ref{sec-dijkstra-versoes-fibonacci}, o heap de Fibonacci possui tempo computacional, aplicado ao algoritmo de Dijkstra, de $O(|V|\lg |V| + |E|)$ enquanto o heap binário possui $O(|E| \lg |V|)$.  Como para todas as instâncias rodadas $|E| > |V|$ (vide tabela \ref{tbl-dijkstra-configuracao}), era de se esperar do ponto de vista teórico que a heap de Fibonacci apresentasse tempos mais rápidos do que o Heap Binário.

Porém, conforme também constatado por \citeonline{larkin2014back}, a aplicação prática das estruturas de dados nem sempre corresponde a esperada descrita na teoria. \citeonline{larkin2014back} mostram que estrutura de dados heaps baseadas em vetor são, na prática, mais eficientes do que a Heap de Fibonacci (vide referência para mais detalhes). É o que os testes realizados por este trabalho também constatam.

\subsection{Conclusões}
\label{sec-dijkstra-conclusoes}
Conforme demonstramos pelos experimentos descritos nesta seção, o algoritmo de Dijkstra que obteve o melhor resultado foi o Heap Binário, sendo mais rápido do que o próprio Heap de Fibonacci que teoricamente deveria ser mais rápido. O Dijkstra canônico obteve tempos que para aplicações como sistema de ponto global (em inglês, GPS) é indesejado, sendo sua implementação interessante apenas para fins de aprendizado e entendimento do algoritmo.

Em termos de implementação, sem dúvida o mais complicado para se implementar foi o Heap de Fibonacci devido a sua própria estrutura que contém uma lista circular duplamente encadeada (a lista de raízes), e pelas funções de restruturação da estrutura que possui muitas movimentações de nodos e tratamento de casos de desvio de condição.

Com isso, colocando em termos práticos, das formas de implementação apresentadas e testadas, a que melhor se sobressai é o Heap Binário que não só foi melhor no tempo dentre outros, como sua implementação é simples.

\section{Algoritmo A*}
\label{sec-aestrela-experimentos}

Para os experimentos computacionais serão utilizados as mesmas instâncias descritas em subseção \ref{sec-dijkstra-experimentos}. Serão comparados quatro versões de algoritmos: o algoritmo de Dijkstra descrito no capítulo \ref{sec-dijkstra}, o algoritmo de Dijkstra adaptado onde o algoritmo é parado assim que se é explorado o vértice objetivo, o algoritmo A* onde se utiliza a heurística admissível distância euclidiana e o algoritmo A* onde se utiliza a heurística não-admissível distância Manhattan, todas sumarizadas na tabela \ref{tbl-aestrela-instancias}.

\begin{table}[H]
\caption{Descrição das versões a serem testadas neste capítulo.}
\label{tbl-aestrela-instancias}
\centering
\begin{adjustbox}{max width=\textwidth}
\begin{tabular}{|c|c|}
\hline 
\textbf{Nome Instância} & \textbf{Descrição} \\ 
\hline 
Dijkstra & Versão de Dijsktra conforme descrito no capítulo \ref{sec-dijkstra} \\ 
\hline 
Dijkstra Adaptado & Versão de Dijkstra adaptado para parar quando o vértice destino é encontrado \\ 
\hline 
Algoritmo A* & Algoritmo A* utilizando a distância euclidiana \\ 
\hline 
Algoritmo A* Manhattan & Algoritmo A* utilizando a distância Manhattan \\ 
\hline 
\end{tabular} 
\end{adjustbox}
\end{table}

Será rodado dez vezes cada algoritmo para cada instância da subseção \ref{sec-dijkstra-experimentos} em que para cada rodada, será afixado o vértice origem como o sendo de valor de identificação "0" e terá como vértice destino um vértice escolhido aleatoriamente, sendo que não haverá repetição de vértices\footnote{Para o caso do algoritmo de Dijkstra especificado na primeira linha da tabela \ref{tbl-aestrela-instancias}, não será estabelecido um vértice destino já que o algoritmo calcula a melhor rota para todos os vértices do grafo.} (ou seja, supomos que o vértice de valor de identificação "180" tenha sido escolhido na primeira rodada. Esse vértice não será escolhido como destino nas demais 9 rodadas. Caso esse vértice seja "sorteado" na próxima iteração, um novo vértice será escolhido aleatoriamente). Nessas dez rodadas será verificado o tempo médio de execução, o número médio de vértices abertos por cada versão e para o algoritmo A* com a heurística Manhattan, será verificado a qualidade da solução.

Para todas as versões será utilizada a estrutura de dados Heap Binário (descrito na subseção \ref{sec-dijkstra-versoes-heap}) pois conforme mostrado no capítulo \ref{sec-dijkstra}, foi a estrutura que melhor se sobressaiu entre as outras em termos de tempo computacional.

\subsection{Resultados obtidos}
\label{sec-aestrela-instancias-resultados}

Os resultados dos testes computacionais descritos anteriormente podem ser vistos nas tabelas \ref{tbl-aestrela-resultados-tempo}, \ref{tbl-aestrela-resultados-qualidadesolucao} e \ref{tbl-aestrela-resultados-nva}.

\begin{table}[H]
\caption{Tempo médio obtido pelos métodos descritos na tabela \ref{tbl-aestrela-instancias} (tempo em milissegundos).}
\label{tbl-aestrela-resultados-tempo}
\centering
\begin{adjustbox}{max width=\textwidth}
\begin{tabular}{|c|c|c|c|c|}
\hline
\textbf{Nome Instância} & \textbf{Dijkstra} & \textbf{Dijkstra Adaptado} & \textbf{Algoritmo A*} & \textbf{Algoritmo A* Manhattan} \\ \hline
USA-road-d.NY.gr        & 236                           & 109                                    & 29                      & 14                                \\ \hline
USA-road-d.BAY.gr       & 321                           & 187                                    & 54                      & 23                                \\ \hline
USA-road-d.COL.gr       & 494                           & 229                                    & 117                     & 27                                \\ \hline
USA-road-d.FLA.gr       & 2858                          & 1026                                   & 886                     & 101                               \\ \hline
\end{tabular} 
\end{adjustbox}
\end{table}

\begin{table}[H]
\caption{Diferença média de solução obtida pelo A* Manhattan com relação a solução ótima.}
\label{tbl-aestrela-resultados-qualidadesolucao}
\centering
\begin{adjustbox}{max width=\textwidth}
\begin{tabular}{|c|c|}
\hline
\textbf{Nome Instância} & \textbf{Qualidade Solução} \\ \hline
USA-road-d.NY.gr        & 5\%                        \\ \hline
USA-road-d.BAY.gr       & 4\%                        \\ \hline
USA-road-d.COL.gr       & 3\%                        \\ \hline
USA-road-d.FLA.gr       & 3\%                        \\ \hline
\end{tabular} 
\end{adjustbox}
\end{table}

\begin{table}[H]
\caption{Número de Vértices Abertos (NVA) médio por cada método.}
\label{tbl-aestrela-resultados-nva}
\centering
\begin{adjustbox}{max width=\textwidth}
\begin{tabular}{|c|c|c|c|}
\hline
\textbf{Nome Instância} & \textbf{NVA Dijkstra Adptado} & \textbf{NVA A*} & \textbf{NVA A* Manhattan} \\ \hline
USA-road-d.NY.gr        & 140501                        & 31995           & 9079                      \\ \hline
USA-road-d.BAY.gr       & 225754                        & 53206           & 17044                     \\ \hline
USA-road-d.COL.gr       & 242674                        & 108343          & 21577                     \\ \hline
USA-road-d.FLA.gr       & 580010                        & 349201          & 79077                     \\ \hline
\end{tabular} 
\end{adjustbox}
\end{table}

\subsection{Análise dos resultados}
\label{sec-aestrela-resultados-analise}
Como apontado pelos testes realizados, o algoritmo A* teve um desempenho computacional melhor do que o algoritmo Dijkstra, inclusive sobre o Dijkstra Adaptado (descrito anteriormente). Podemos ver que esse resultado está diretamente ligado ao número de vértices abertos por cada algoritmo (tabela \ref{tbl-aestrela-resultados-nva}). Aqueles que abriram mais vértices, obtiveram um tempo computacional maior. Isso já esperado, já que o algoritmo teve que processar mais etapas até que sua condição de parada fosse encontrada.

Dos quatro métodos testados, o que obteve menor tempo computacional foi o algoritmo A* aplicando a heurística não-admissível Distância Manhattan. Isso se deve ao fato de o cálculo da distância (que é realizado em tempo de execução) ser mais simples do que o empregado pela distância euclidiana (que envolve radiação e exponenciação). Porém esse resultado possui um "preço a ser pago" que é, conforme descrito na subseção \ref{sec-aestrela-algoritmo-heuristica}, a não garantia do menor caminho entre os vértices pesquisados. Mas, conforme demonstra a tabela \ref{tbl-aestrela-resultados-qualidadesolucao}, a diferença média entre as soluções encontradas e suas respectivas soluções ótimas giram em torno de 4\%, o que pode ser considerado como um bom resultado.

\subsection{Conclusões}
\label{sec-aestrela-conclusao}
O algoritmo A* demonstra ser um ótimo algoritmo para o cálculo de menor caminho entre dois vértices (origem e destino), superando em tempo computacional o algoritmo de Dijsktra (mas que retorna a menor rota para todos os demais vértices), sendo mais indicado para esse tipo de cálculo. Sua implementação é simples e é praticamente uma adaptação do algoritmo de Dijsktra.

Com relação as heurísticas, para a garantia do melhor caminho como o algoritmo de Dijkstra o faz, é obrigatório o uso de uma heurística admissível, mesmo que isso tenha um impacto negativo no tempo com relação ao uso de outras heurísticas (as não-admissíveis). Porém, conforme mostrado na subseção ref{sec-aestrela-instancias-resultados}, a diferença entre as soluções ótimas e obtidos pelo método Manhattan giraram em torno de 4\%, o que é um bom resultado para quem deseja menor tempo computacional e não possui a obrigatoriedade do menor caminho.

\section{Algoritmos Dinâmicos}
\label{sec-experimentos-dinamicos}
Para o capítulo de algoritmo dinâmicos serão rodados dois tipos testes, um para cada algoritmo. Serão utilizadas as mesmas instâncias descritas na seção \ref{sec-dijkstra-experimentos}.

\subsection{ARA*}
\label{sec-experimentos-dinamicos-ara}

Para os testes com o algoritmo ARA*, será comparado o desempenho da utilização da heurística inflada com relação ao desempenho do algoritmo de Dijkstra. Será afixado um conjunto arbitrário de $\epsilon$ e será medido quanto tempo o algoritmo ARA* acha uma solução (mesmo não sendo a ótima) para aquele determinado $\epsilon$ e comparar com o tempo que o algoritmo A* acha uma solução ótima, além do número médio de vértices abertos por cada algoritmo. Será utilizada a mesma forma de bateria de testes descritas na seção \ref{sec-aestrela-experimentos} em que serão escolhidos 10 vértices diferentes aleatórios como vértice destino e 0 como vértice origem e depois disso, tirado a média dos tempo total gasto para achar a rota.

\subsubsection{Resultados obtidos}
\label{sec-experimentos-dinamicos-ara-resultados}

Os resultados dos experimentos estão descritos a seguir na tabela \ref{tbl-experimentos-dinamicos-ara}. O tempo é descrito em nanosegundos (ns). Observe também que o valor do tempo médio do algoritmo A* não varia dentro da mesma instância. Isso é devido a ser utilizado o resultado único para cada estância, já que o valor de $\epsilon$ não é utilizado pelo A*.

\begin{table}[H]
\caption{Resultado testes ARA*.}
\label{tbl-experimentos-dinamicos-ara}
\centering
\begin{adjustbox}{max width=\textwidth}
\begin{tabular}{|c|c|c|c|c|c|c|}
\hline
\textbf{Nome Instância}            & \multicolumn{1}{l|}{\textbf{Epsisolon}} & \multicolumn{1}{l|}{\textbf{NVA A*}} & \multicolumn{1}{l|}{\textbf{Tempo médio A*}} & \multicolumn{1}{l|}{\textbf{NVA ARA*}} & \multicolumn{1}{l|}{\textbf{Tempo médio ARA*}} & \multicolumn{1}{l|}{\textbf{Ganho de tempo com realção ao A*}} \\ \hline
\multirow{7}{*}{USA-road-d.NY.gr}  & 4.0                                     & 29394                                & 26900000                                     & 252                                    & 980000                                         & 2.700\%                                                        \\ \cline{2-7} 
                                   & 3.5                                     & 29394                                & 26900000                                     & 257                                    & 320000                                         & 8.400\%                                                        \\ \cline{2-7} 
                                   & 3.0                                     & 29394                                & 26900000                                     & 265                                    & 820000                                         & 3.200\%                                                        \\ \cline{2-7} 
                                   & 2.5                                     & 29394                                & 26900000                                     & 281                                    & 840000                                         & 3.200\%                                                        \\ \cline{2-7} 
                                   & 2.0                                     & 29394                                & 26900000                                     & 296                                    & 560000                                         & 4.800\%                                                        \\ \cline{2-7} 
                                   & 1.5                                     & 29394                                & 26900000                                     & 360                                    & 940000                                         & 2.800\%                                                        \\ \cline{2-7} 
                                   & 1.0                                     & 29394                                & 26900000                                     & 4825                                   & 3140000                                        & 800\%                                                          \\ \hline
\multirow{7}{*}{USA-road-d.BAY.gr} & 4.0                                     & 43896                                & 45980000                                     & 239                                    & 1180000                                        & 3.800\%                                                        \\ \cline{2-7} 
                                   & 3.5                                     & 43896                                & 45980000                                     & 236                                    & 800000                                         & 5.700\%                                                        \\ \cline{2-7} 
                                   & 3.0                                     & 43896                                & 45980000                                     & 224                                    & 800000                                         & 5.700\%                                                        \\ \cline{2-7} 
                                   & 2.5                                     & 43896                                & 45980000                                     & 233                                    & 620000                                         & 7.400\%                                                        \\ \cline{2-7} 
                                   & 2.0                                     & 43896                                & 45980000                                     & 252                                    & 1040000                                        & 4.400\%                                                        \\ \cline{2-7} 
                                   & 1.5                                     & 43896                                & 45980000                                     & 299                                    & 740000                                         & 6.200\%                                                        \\ \cline{2-7} 
                                   & 1.0                                     & 43896                                & 45980000                                     & 10867                                  & 7920000                                        & 500\%                                                          \\ \hline
\multirow{7}{*}{USA-road-d.COL.gr} & 4.0                                     & 69991                                & 68140000                                     & 99                                     & 580000                                         & 11.700\%                                                       \\ \cline{2-7} 
                                   & 3.5                                     & 69991                                & 68140000                                     & 93                                     & 860000                                         & 7.900\%                                                        \\ \cline{2-7} 
                                   & 3.0                                     & 69991                                & 68140000                                     & 92                                     & 780000                                         & 8.700\%                                                        \\ \cline{2-7} 
                                   & 2.5                                     & 69991                                & 68140000                                     & 92                                     & 720000                                         & 9.400\%                                                        \\ \cline{2-7} 
                                   & 2.0                                     & 69991                                & 68140000                                     & 90                                     & 500000                                         & 13.600\%                                                       \\ \cline{2-7} 
                                   & 1.5                                     & 69991                                & 68140000                                     & 172                                    & 580000                                         & 11.700\%                                                       \\ \cline{2-7} 
                                   & 1.0                                     & 69991                                & 68140000                                     & 2080                                   & 1720000                                        & 3.900\%                                                        \\ \hline
\multirow{7}{*}{USA-road-d.FLA.gr} & 4.0                                     & 255312                               & 711120000                                    & 808                                    & 3500000                                        & 20.300\%                                                       \\ \cline{2-7} 
                                   & 3.5                                     & 255312                               & 711120000                                    & 840                                    & 2820000                                        & 25.200\%                                                       \\ \cline{2-7} 
                                   & 3.0                                     & 255312                               & 711120000                                    & 913                                    & 2460000                                        & 28.900\%                                                       \\ \cline{2-7} 
                                   & 2.5                                     & 255312                               & 711120000                                    & 596                                    & 2220000                                        & 32.000\%                                                       \\ \cline{2-7} 
                                   & 2.0                                     & 255312                               & 711120000                                    & 524                                    & 1940000                                        & 36.600\%                                                       \\ \cline{2-7} 
                                   & 1.5                                     & 255312                               & 711120000                                    & 687                                    & 2060000                                        & 34.500\%                                                       \\ \cline{2-7} 
                                   & 1.0                                     & 255312                               & 711120000                                    & 18909                                  & 14360000                                       & 4.900\%                                                        \\ \hline
\end{tabular}
\end{adjustbox}
\end{table}

\subsubsection{Análise dos resultados}
\label{sec-experimentos-dinamicos-ara-analise}




% ==============================================================================
% TCC - Nome do Aluno
% Capítulo 5 - Testes Computacionais
% ==============================================================================
\chapter{Considerações Finais}
\label{sec-conclusao}



%%% Páginas finais do documento: bibliografia e anexos. %%%

% Finaliza a parte no bookmark do PDF para que se inicie o bookmark na raiz e adiciona espaço de parte no sumário.
\phantompart

% Marca o início dos elementos pós-textuais.
\postextual

% Referências bibliográficas
\bibliography{bibliografia}


% Apêndices.



% Índice remissivo.
\phantompart
\printindex

% Fim do documento.
\end{document}
