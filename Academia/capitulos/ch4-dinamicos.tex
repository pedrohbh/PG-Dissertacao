% ==============================================================================
% TCC - Nome do Aluno
% Capítulo 3 - Projeto Arquitetural e Implementação
% ==============================================================================
\chapter{Algoritmos de menor caminho em grafos dinâmicos}
\label{sec-dinamicos}

\section{Grafos Dinâmicos: Definição}
\label{sec-dinamicos-grafos}
Os algoritmos apresentados até agora tratam apenas de grafos estáticos, ou seja, grafos em que o peso de suas arestas não mudam. Mas em situações reais podemos ter casos em que a modelagem feita por grafos requer que os pesos de suas arestas variem com o tempo. Esses grafos são ditos \textbf{dinâmicos}. Exemplos de modelagem por grafos dinâmicos são o sistema de tempo real de trânsito em que os pesos das arestas correspondem ao tempo médio para percorrer um determinado trecho e esse tempo está diretamente ligado ao trânsito local em uma determinada hora; e o fluxo uma rede interna de computadores em que as arestas representam o caminho entre roteadores e os seus pesos correspondem ao uso desta linha, ou seja, o quão congestionada está.

Para calcularmos o menor caminho entre dois vértices em grafos dinâmicos, poderíamos utilizar os algoritmos de Dijsktra e A* (discutidos nos capítulos \ref{sec-dijkstra} e \ref{sec-aestrela})  para acharmos a rota, e quando houver uma detecção de mudança do peso de arestas, recalcularíamos novamente o trajeto reutilizando esses algoritmos. Em geral desejamos (ou necessitamos) que esse recálculo seja mais rápido. Para esse fim existem os algoritmos dinâmicos que calculam uma solução rápida, porém não garantidamente ótima (elas são ditas sub-ótimas).

Exemplos de algoritmos são o D* \cite{stentz1994optimal}, D* Lite \cite{koenig2002d} e o AD* \cite{likhachev2008anytime}.

Neste trabalho serão discutidos os algoritmos ARA* e AD* \cite{likhachev2008anytime}.
\section{Algoritmo ARA*}
\label{sec-dinamicos-ara}

O algoritmo ARA* proposto em \citeonline{likhachev2008anytime}, está descrito nos três algoritmos a seguir.

\begin{algorithm}[H]
\SetAlgoLined
%\KwResult{Write here the result}
%\Entrada{weighted simple digraph, vertex first, vertex goal}
\While{key($s_{goal}$) > $min_{s \in OPEN(key(s))}$}
{
	remove s with smallest key(s) from OPEN\;
	v(s) = g(s)\; CLOSED = CLOSED $\cup$ {s}\;
	\ForEach{successor s' of s}
	{
		\If{s' was never visited by ARA* before then}
		{
			v(s') $\leftarrow$ $\infty$\;
			g(s') $\leftarrow$ $\infty$\;
		}
		\If{g(s') > g(s) + c(s, s')}
		{
			g(s') $\leftarrow$ g(s) + c(s, s')\;
			\eIf{s' $\notin$ CLOSED}
			{
				insert/update s' in OPEN with key(s')\;
			}
			{
				insert s' into INCONS\;
			}
		}
	}
}
\caption{Algoritmo ARA* - função de cálculo de caminho ComputePath().}
\end{algorithm}


%\begin{lstlisting}[mathescape, label=lst-dinamicos-ara-computepath, caption=Algoritmo ARA* - função de cálculo de caminho, float=htpb]
%procedure ComputePath()
%	while(key($s_{goal}$) > $min_{s \in OPEN(key(s))}$)
%		remove s with smallest key(s) from OPEN;
%		v(s) = g(s); CLOSED = CLOSED $\cup$ {s};
%		for each successor s' of s
%			if s' was never visited by ARA* before then
%				v(s') = g(s') = $\infty$;
%			if g(s') > g(s) + c(s, s')
%				g(s') = g(s) + c(s, s');
%				if s' $\notin$ CLOSED
%					insert/update s' in OPEN with key(s');
%				else
%					insert s' into INCONS;
%\end{lstlisting}

\begin{algorithm}[H]
\SetAlgoLined
%\KwResult{Write here the result}
\Entrada{vertex s}
return g(s) + $\epsilon$ * h(s)\;
\caption{Algoritmo ARA* - função da chave ordenadora da fila de prioridades key(s).}
\end{algorithm}

%\begin{lstlisting}[mathescape, label=lst-dinamicos-ara-key, caption=Algoritmo ARA* - função da chave ordenadora da fila de prioridades, float=htpb]
%procedure key(s)
%	return g(s) + $\epsilon$ * h(s);
%\end{lstlisting}
\newpage
\begin{algorithm}[H]
\SetAlgoLined
%\KwResult{Write here the result}
%\Entrada{weighted simple digraph, vertex first, vertex goal}
g($s_{goal}$) $\leftarrow$ v($s_{goal}$) $\leftarrow$ $\infty$; v($s_{start}$) $\leftarrow$ $\infty$\;
g($s_{start}$) $\leftarrow$ 0; OPEN $\leftarrow$ CLOSED $\leftarrow$ INCONS $\leftarrow$ $\emptyset$\;
insert $s_{start}$ into OPEN with key($s_{start}$)\;
ComputePath()\;
publish current $\epsilon$-suboptimal solution\;
\While{$\epsilon > 1$}
{
	decrease $\epsilon$\;
	Move states from INCONS into OPEN\;
	Update the priorities for all s $\in$ OPEN according to key(s)\;
	CLOSED $\leftarrow$ $\emptyset$\;
	ComputePath()\;
	publish current $\epsilon$-suboptimal solution\;
}
\caption{Algoritmo ARA* - função principal.}
\end{algorithm}

%\begin{lstlisting}[mathescape, label=lst-dinamicos-ara-main, caption=Algoritmo ARA* - função principal, float=htpb]
%procedure Main()
%	g($s_{goal}$) = v($s_{goal}$) = $\infty$; v($s_{start}$) = $\infty$;
%	g($s_{start}$) = 0; OPEN = CLOSED = INCONS = $\emptyset$;
%	insert $s_{start}$ into OPEN with key($s_{start}$);
%	ComputePath();
%	publish current $\epsilon$-suboptimal solution;
%	while $\epsilon > 1$
%		decrease $\epsilon$;
%		Move states from INCONS into OPEN
%		Update the priorities for all s $\in$ OPEN according to key(s);
%		CLOSED = $\emptyset$;
%		ComputePath();
%		publish current $\epsilon$-suboptimal solution;
%\end{lstlisting}

A função principal é descrita no terceiro algoritmo. Temos as variáveis g(s) e v(s) que correspondem ao valor da distância real calculada da origem $s_{start}$ até o vértice s, porém v(s) representa a distância calculada pelo algoritmo na busca anterior, enquanto g(s) é a distância da busca atual. O valor de v(s), apesar de estar descrito no algoritmo conforme a literatura de origem, não é utilizado pelo algoritmo. Ele só é utilizado no próximo algoritmo, AD*, que está descrito no mesmo artigo do ARA* e foi do critério de seus autores originais colocar essa variável em sua descrição (portanto será ignorado na explicação que se segue). O valor de g($s_{goal}$) (distância real calculada da origem ao vértice destino) é atribuído como $\infty$ (infinito)\footnote{Vide nota de rodapé da seção \ref{sec-dijkstra-algoritmo}.}, enquanto que o valor de g($s_{start}$) é atribuído com o valor 0, pois a distância real é calculada a partir da origem. Em seguida são criados os conjuntos ``OPEN'', ``CLOSED'' e ``INCONS'', correspondendo respectivamente ao conjunto dos ``ABERTOS'', ``FECHADOS'' e ``INCONSISTENTES''. O vértice $s_{start}$ é inserido na fila de prioridades do conjunto OPEN utilizando a função de chave ordenadora descrito no segundo algoritmo. Essa função utiliza a estratégia da heurística inflada (descrito a seguir, na subseção \ref{sec-dinamicos-ara-consideracoes}).

A função de cálculo de caminho é então acionada pela função principal (algoritmo \ref{lst-dinamicos-ara-computepath}). Nesta função, o processo iterativo ocorre enquanto o valor da chave de $s_{goal}$ for maior do que o valor da chave do termo mínimo da fila de prioridades. Enquanto isso for verdade, o vértice s, que corresponde ao valor do vértice com menor chave de OPEN, é removido deste e adicionado ao conjunto CLOSED. Para cada s' (que representa um vértice do conjunto de vértices adjacentes de s) é verificado se o mesmo já foi visitado pelo algoritmo e em caso negativo, o seu valor de g(s') é atribuído como $\infty$ (infinito). Em seguida, é verificado se a distância calculada até o momento para s', g(s'), é maior do que a soma de g(s) com o peso da aresta entre s e s'. Se caso positivo, g(s') é atualizado com esse novo valor. Em seguida é verificado se o vértice s' não pertence ao conjunto dos fechados e em caso positivo, ele é adicionado/atualizado no conjunto OPEN de acordo com a função de chave ordenadora. Caso contrário, ele é adicionado ao conjunto dos INCONS.

Após o cálculo do caminho e de acordo com função principal (algoritmo \ref{lst-dinamicos-ara-main}), a solução de fator $\epsilon$ é sub-ótima (o motivo da solução ser sub-ótima é explicado na subseção \ref{sec-dinamicos-ara-consideracoes}) é apresentada pelo algoritmo. A partir daí começa o processo iterativo em que se verifica se o valor de $\epsilon$ é maior que 1. Em caso positivo, o valor de $\epsilon$ é decrescido por um fator de corte, estipulado como parâmetro de entrada. Os vértices pertencentes a INCONS são movidos para OPEN e todas as chaves da fila de prioridades são atualizadas considerando o novo valor de $\epsilon$. O conjunto CLOSED é esvaziado e a função de cálculo de caminho é acionada novamente e seu respectivo resultado é apresentado.

\subsection{Heurística inflada e considerações sobre o algoritmo}
\label{sec-dinamicos-ara-consideracoes}

A principal diferença entre o algoritmo ARA* e o A* está na utilização da estratégia da heurística inflada. Ela consiste em utilizar a mesma função de ordenação de chaves da fila de prioridades do algoritmo A*, com a exceção de que o valor heurístico é multiplicado por um fator $\epsilon$. Isso faz com que menos vértices sejam visitados, já que os menos ``promissores'' serão posicionados mais para o final da fila de prioridades enquanto que os mais ``promissores'' serão posicionados mais para frente, e com isso é provável que um caminho até o vértice destino, $s_{goal}$, seja encontrado mais rápido do que pelo algoritmo A* (já que menos vértices deverão ser visitados). Entretanto, ao se utilizar esta técnica perdemos a garantia do resultado ótimo do algoritmo (algo semelhante ao que ocorre quando não utilizamos heurísticas não-admissíveis. Vide subseção \ref{sec-aestrela-algoritmo-heuristica}).

Porém, uma grande vantagem de se utilizar essa estratégia é que teremos um limite superior para a solução encontrada. Considere que o custo ótimo de um determinado caminho seja C*. Se utilizarmos a função g(s) + $\epsilon$ * h(s), com o valor de $\epsilon > 1$, então há a garantia de que a nova solução C, C* $\leq$ C $\leq \epsilon$ x C*. Com isso, se $\epsilon$ = 1, a solução encontrada será garantidamente ótima \cite{moura2010estudo}.

Portanto a ideia principal é achar uma solução rápida, porém não-garantidamente ótima, atribuindo um valor $\epsilon$ maior do que 1, e caso a aplicação permita o uso de um tempo maior para aprimorarmos o resultado\footnote{Por exemplo, uma determinada aplicação necessita que se ache um caminho em no máximo 2 segundos. Se o algoritmo achar uma solução sub-ótima em 3 ms, dispomos de mais 1997 ms para podermos aprimorá-la.}, diminuímos o valor de $\epsilon$. E se o tempo ainda permitir, diminuímos até termos $\epsilon$ = 1, em que a solução será garantidamente ótima. 

Outra estratégia utilizada para ganhar tempo, a cada novo valor de $\epsilon$ em que a função para cálculo de caminho é acionada (algoritmo \ref{lst-dinamicos-ara-computepath}), não é necessário recalcular todos os vértices visitados anteriormente, já que os mesmos permanecem na lista de abertos, tendo apenas suas chaves atualizadas com o novo valor de $\epsilon$ e consequentemente a ordem dos vértices também é reconfigurada de acordo com esses novos valores. Além disso, caso haja um melhor caminho encontrado para um vértice que pertence aos fechados, este são colocados em INCONS dos quais serão enviados para OPEN para serem recalculados na próxima busca. A busca se refere ao número da iteração da invocação da função de cálculo de caminho, contido no algoritmo \ref{lst-dinamicos-ara-computepath}, e consequentemente ao valor de $\epsilon$ a ele atribuído.

As figuras \ref{fig-ara-exemplo3}, \ref{fig-ara-exemplo1} e \ref{fig-ara-exemplo2} mostram a estratégia da heurística inflada aplicada. A figura \ref{fig-ara-exemplo3} mostra o grafo onde são aplicados os algoritmos A* e o ARA*, sendo determinado como o vértice origem o ``0'' e o vértice destino o ``4''. Os valores das heurísticas utilizadas correspondem ao custo da distância ótima de cada vértice ao vértice destino ``4''. A figura \ref{fig-ara-exemplo1} mostra a exploração do grafo pelo algoritmo A* (os vértices em vermelho representam os que são escolhidos pelo algoritmo a cada passo), enquanto que a figura \ref{fig-ara-exemplo2} mostra a exploração do grafo pelo algoritmo ARA*, tendo o valor de $\epsilon = 2$ (novamente os vértices em vermelho representam os que são escolhidos pelo algoritmo a cada passo). Podemos ver que o efeito da heurística inflada ``poda'' ainda mais os vértices a serem visitados e encontra uma solução que não é a ótima.

A figura \ref{fig-ara-exemplo}, contida em \citeonline{likhachev2008anytime}, mostra a estratégia de diminuir o fator $\epsilon$ e as respectivas soluções encontradas para a configuração proposta pela figura.
%A figura \ref{fig-ara-exemplo}, contida em \citeonline{likhachev2008anytime}, mostra a aplicação da estratégia da heurística inflada com seus respectivos resultados encontrados pelo algoritmo.
\newpage
\begin{figure}[H]
\centering
\includegraphics[width=.70\textwidth]{figuras/ARA-exemplo3} 
\caption{Exemplo de grafo onde são aplicados os algoritmos A* e ARA*.}
\label{fig-ara-exemplo3}
\end{figure}

\begin{figure}[H]
\centering
\includegraphics[width=.70\textwidth]{figuras/ARA-exemplo1} 
\caption{Exploração do grafo da figura \ref{fig-ara-exemplo3} pelo algoritmo A*.}
\label{fig-ara-exemplo1}
\end{figure}

\begin{figure}[H]
\centering
\includegraphics[width=.70\textwidth]{figuras/ARA-exemplo2} 
\caption{Exploração do grafo da figura \ref{fig-ara-exemplo3} pelo algoritmo ARA*, tendo o valor $\epsilon = 2$.}
\label{fig-ara-exemplo2}
\end{figure}

\begin{figure}[H]
\centering
\includegraphics[width=.80\textwidth]{figuras/ara-3} 
\caption{Exemplo de aplicação da estratégia de diminuição do valor de $\epsilon$.}
\label{fig-ara-exemplo}
\end{figure}



\section{Algoritmo AD*}
\label{sec-dinamicos-ad}

O algoritmo ARA* calcula um caminho para grafos estáticos e dinâmicos. Porém, para o caso dos grafos dinâmicos, o ARA* contempla somente o tipo de mudança para quando o peso das arestas diminuem e não para quando elas aumentam \cite{moura2010estudo}. Por isso o algoritmo AD* foi proposto.

Esse algoritmo, também proposto em \citeonline{likhachev2008anytime}, está descrito nos quatro algoritmos a seguir.

\begin{algorithm}[H]
\SetAlgoLined
%\KwResult{Write here the result}
\Entrada{vertex s}
\eIf{$v(s) \neq g(s)$}
{
	\eIf{s $\notin$ CLOSED}
	{
		insert/update s in OPEN with key(s)\;
	}
	{
		\If{s $\notin$ INCONS}
		{
			insert s in INCONS\;
		}
	}
}
{
	\eIf{s $\in$ OPEN}
	{
		remove s from OPEN\;
	}
	{
		\If{s $\in$ INCONS}
		{
			remove s from INCONS\;
		}
	}
}
\caption{Algoritmo AD* - função para determinar o conjunto ao qual vértice pertencerá, UpdateSetmembership(s).}
\end{algorithm}

%\begin{lstlisting}[mathescape, label=lst-dinamicos-ad-set, caption=Algoritmo AD* - função para determinar o conjunto ao qual vértice pertencerá, float=htpb]
%procedure UpdateSetMembership(s)
%	if ($v(s) \neq g(s)$)
%		if (s $\notin$ CLOSED) insert/update s in OPEN with key(s);
%		else if (s $\notin$ INCONS) insert s in INCONS;
%	else
%		if (s $\in$ OPEN) remove s from OPEN;
%		else if (s $\in$ INCONS) remove s from INCONS;
%\end{lstlisting}
\newpage
\begin{algorithm}[H]
\SetAlgoLined
%\KwResult{Write here the result}
%\Entrada{vertex s}
\While{key($s_{goal}$) > $min_{s \in OPEN(key(s))}$ OR v($s_{goal}$) < g($s_{goal}$)}
{
	remove s with smallest key(s) from OPEN\;
	\eIf{v(s) > g(s)}
	{
		v(s) $\leftarrow$ g(s); CLOSED $\leftarrow$ CLOSED $\cup$ {s}\;
		\ForEach{successor s' of s}
		{
			\If{s' was never visited by AD* before then}
			{
				v(s') $\leftarrow$ g(s') $\leftarrow$ $\infty$;bp(s') $\leftarrow$ null\;
			}
			\If{g(s') > g(s) + c(s, s')}
			{
				g(s') $\leftarrow$ g(s) + c(s, s')\;
				bp(s') $\leftarrow$ s\;
				g(s') $\leftarrow$ g(bp(s')) + c(bp(s'),s')\; UpdateSetMembership(s')\;
			}
		}
	}
	{
		v(s) $\leftarrow$ $\infty$\; UpdateSetMembership(s)\;	
		\ForEach{successor s' of s}
		{
			\If{s' was never visited by AD* before then}
			{
				v(s') $\leftarrow$ g(s') $\leftarrow$ $\infty$;bp(s') $\leftarrow$ null\;
			}
			\If{bp(s') = s}
			{
				bp(s') $\leftarrow$ $argmin_{s'' \in pred(s')}$ v(s'') + c(s'',s')\;
				g(s') $\leftarrow$ v(bp(s')) + c(bp(s'),s')\; UpdateSetMembership(s')\;
			}
		}
	}
}
\caption{Algoritmo AD* - função de cálculo de caminho ComputePath().}
\end{algorithm}

%\begin{lstlisting}[mathescape, label=lst-dinamicos-ad-computepath, caption=Algoritmo AD* - função de cálculo de caminho, float=htpb]
%procedure ComputePath()
%	while(key($s_{goal}$) > $min_{s \in OPEN(key(s))}$ OR v($s_{goal}$) < g($s_{goal}$)) 
%		remove s with smallest key(s) from OPEN;
%		if v(s) > g(s)
%			v(s) = g(s); CLOSED = CLOSED $\cup$ {s};
%			for each successor s' of s
%				if s' was never visited by AD* before then
%					v(s') = g(s') = $\infty$;bp(s') = null;
%				if g(s') > g(s) + c(s, s')
%					g(s') = g(s) + c(s, s');
%					bp(s') = s;
%					g(s') = g(bp(s')) + c(bp(s'),s'); UpdateSetMembership(s');
%		else
%			v(s) = $\infty$; UpdateSetMembership(s);
%			for each successor s' of s
%			if s' was never visited by AD* before then
%				v(s') = g(s') = $\infty$;bp(s') = null;
%			if bp(s') = s
%				bp(s') = $argmin_{s'' \in pred(s')}$ v(s'') + c(s'',s');
%				g(s') = v(bp(s')) + c(bp(s'),s'); UpdateSetMembership(s');
%\end{lstlisting}

\begin{algorithm}[H]
\SetAlgoLined
%\KwResult{Write here the result}
\Entrada{vertex s}
\eIf{v(s) $\geq$ g(s)}
{
	return [g(s) + $\epsilon$ * h(s); g(s)]\;
}
{
	return [v(s) + h(s); v(s)]\;
}
\caption{Algoritmo AD* - função da chave ordenadora da fila de prioridades key(s).}
\end{algorithm}

%\begin{lstlisting}[mathescape, label=lst-dinamicos-ad-key, caption=Algoritmo AD* - função da chave ordenadora da fila de prioridades, float=htpb]
%procedure key(s)
%	if (v(s) $\geq$ g(s))
%		return [g(s) + $\epsilon$ * h(s); g(s)];
%	else
%		return [v(s) + h(s); v(s)];
%\end{lstlisting}
\newpage
\begin{algorithm}[H]
\SetAlgoLined
%\KwResult{Write here the result}
%\Entrada{weighted simple digraph, vertex first, vertex goal}
g($s_{goal}$) $\leftarrow$ v($s_{goal}$) $\leftarrow$ $\infty$; v($s_{start}$) $\leftarrow$ $\infty$;bp($s_{goal}$) $\leftarrow$ bp($s_{start}$) $\leftarrow$ null\;
g($s_{start}$) $\leftarrow$ 0; OPEN $\leftarrow$ CLOSED $\leftarrow$ INCONS $\leftarrow$ $\emptyset$; $\epsilon$ $\leftarrow \epsilon_{0}$\;
insert $s_{start}$ into OPEN with key($s_{start}$)\;
\While{ forever }
{
	ComputePath()\;
	publish current $\epsilon$-suboptimal solution\;
	\If{$\epsilon = 1$}
	{
		wait changes in edge costs\;
	}
	\ForEach{directed edges (u,v) with changed edge costs}
	{
		update the edge cost c(u,v)\;
		\If{$v \neq s_{start}$ AND v was visited by AD* before}
		{
			bp(v) $\leftarrow$ $argmin_{s'' \in pred(v)}$ v(s'') + c(s'',v)\;
			g(v) $\leftarrow$ v(bp(v)) + c(bp(v),v); UpdateSetMembership(v)\;
		}
	}
	\eIf{significant edge cost changes were observed}
	{
		increase $\epsilon$ or re-plan from scratch (i.e., re-excute Main function);
	}
	{
		\If{$\epsilon > 1$}
		{
			decrease $\epsilon$\;
		}
	}
		Move states from INCONS into OPEN\;
		Update the priorities for all s $\in$ OPEN according to key(s)\;
		CLOSED $\leftarrow$ $\emptyset$\;
}
		
\caption{Algoritmo AD* - função principal.}
\end{algorithm}


%\begin{lstlisting}[mathescape, label=lst-dinamicos-ad-main, caption=Algoritmo AD* - função principal, float=htpb]
%procedure Main()
%	g($s_{goal}$) = v($s_{goal}$) = $\infty$; v($s_{start}$) = $\infty$;bp($s_{goal}$) = bp($s_{start}$) = null;
%	g($s_{start}$) = 0; OPEN = CLOSED = INCONS = $\emptyset$; $\epsilon = \epsilon_{0}$;
%	insert $s_{start}$ into OPEN with key($s_{start}$);
%	forever
%		ComputePath();
%		publish current $\epsilon$-suboptimal solution;
%		if $\epsilon = 1$
%			wait changes in edge costs;
%		for all directed edges (u,v) with changed edge costs
%			update the edge cost c(u,v);
%			if ( $v \neq s_{start}$ AND v was visited by AD* before)
%				bp(v) = $argmin_{s'' \in pred(v)}$ v(s'') + c(s'',v);
%				g(v) = v(bp(v)) + c(bp(v),v); UpdateSetMembership(v);
%		if significant edge cost changes were observed
%			increase $\epsilon$ or re-plan from scratch (i.e., re-excute Main function);
%		else if $\epsilon > 1$
%			decrease $\epsilon$;
%		Move states from INCONS into OPEN
%		Update the priorities for all s $\in$ OPEN according to key(s);
%		CLOSED = $\emptyset$;
%\end{lstlisting}

O algoritmo é muito semelhante em sua execução ao ARA*, já que o AD* é uma adaptação do mesmo \cite{moura2010estudo}. A função principal (quarto algoritmo) começa iniciando os valores de g($s_{goal}$), v($s_{goal}$) e v($s_{start}$) como $\infty$, bp($s_{start}$) como ponteiro nulo e g($s_{start}$) como 0. Os valores g(s) e v(s) correspondem as distâncias reais calculadas partindo do vértice origem ao vértice s, sendo que o valor g(s) corresponde a distância da busca atual enquanto que o v(s) corresponde ao valor da busca anterior. O valor bp(s) corresponde ao ponteiro para o vértice antecessor de s. Seguindo com o algoritmo, o vértice $s_{start}$ é inserido na fila de prioridades de acordo com  a função de chave ordenadora descrita no terceiro algoritmo. Em seguida, é acionado a função de cálculo de caminho (segundo algoritmo).

Essa função trabalha da seguinte forma: inicialmente se verifica se o valor da chave de $s_{goal}$ é maior do que a menor chave da fila de prioridades (condição semelhante ao do algoritmo ARA*) ou se o valor de v($s_{goal}$) é menor do que o valor de g( $s_{goal}$ ). Caso seja verdade, o processo iterativo é iniciado. O vértice s, que corresponde ao vértice com menor chave na fila de prioridades é removido do conjunto OPEN, e é verificado se o valor de v(s) é maior do que o g(s), ou seja, se o valor da distância calculada da busca anterior é maior do que o valor da busca atual (esse é o caso padrão do algoritmo). Sendo verdade, o algoritmo segue a mesma forma de execução do ARA*, conforme descrito na seção \ref{sec-dinamicos-ara}. A diferença de tratamento ocorre quando a condição inicial não ocorre, e neste caso, o valor de v( s ) é atualizado como $\infty$, e para cada vértice s' adjacente de s é verificado se este já havia sido visitado pelo algoritmo. Caso contrário, os valores de v( s' ), g( s' ) são iniciados como $\infty$ e bp( s') é iniciado como ponteiro nulo. Em seguida, é verificado se bp( s' ) é igual ao vértice s. Em caso positivo, o bp( s' ) é atualizado com o vértice predecessor de s', s'', cujo valor da função v(s'') + c( s'', s') é a menor dentre todos os vértices predecessores de s'. Feito isso, o valor de g( s' ) é atualizado com esse novo valor. Finalmente a função de determinação para qual conjunto pertencerá s' é chamada (função ``UpdateSetMembership'').

A função de determinação ao qual conjunto pertencerá funciona de uma maneira muito simples. Caso os valores de v( s ) e g( s ) sejam diferentes, a função tem comportamento similar ao processo de atribuição de conjunto do ARA*. A diferença ocorre quando esses valores são iguais (ou seja, não houve mudança de valores entre a busca atual e a anterior). O vértice é removido de OPEN caso pertença a ele ou removido de INCONS caso pertença a este. Isso feito para que este vértice não precise ser explorado nesta busca, caso esteja em OPEN, ou na próxima, caso esteja em INCOS.

Voltando ao método principal (quarto algoritmo), a função segue o mesmo padrão da ARA*. A única exceção se dá quando mudanças no grafo são detectadas (mudança dos pesos das arestas). Neste caso, para cada aresta (u,v) mudada, é atribuído ao valor bp( v ), o vértice s'', predecessor de v, cuja função v( s'' ) + c( s'', v) seja mínima. Consequentemente, o valor de g( v ) é atualizado conforme esse valor e a função de atribuição de conjunto é acionada. 