% ==============================================================================
% TCC - Nome do Aluno
% Capítulo 3 - Especificação de Requisitos
% ==============================================================================
\chapter{Algoritmo A*}
\label{sec-aestrela}

\section{O Algoritmo}
\label{sec-aestrela-algoritmo}
O algoritmo A* (lê-se "A estrela") também conhecido como busca A* é um algoritmo de busca informatizada em grafos. Foi proposta originalmente por (buscar referência) em (buscar referência) e pode ser visto como uma adaptação do algoritmo de Dijkstra (apresentado no capítulo \ref{sec-dijkstra}) em que, ao invés de se calcular a melhor rota de um ponto de partida para todos os demais vértices do grafo, se estabelece um vértice origem e busca calcular uma boa rota para um vértice destino (ou mesmo a rota ótima. A garantia do valor ótimo do algoritmo depende de fatores que serão discutidos a seguir), realizando "podas" do caminho de forma que não seja necessário visitar todos os vértices, apenas os mais promissores.

A seguir é apresentado o algoritmo A* conforme demonstrado em (buscar referência).


\subsection{Heurísticas admissíveis e não-admissíveis}
\label{sec-aestrela-algoritmo-heuristica}  

\section{Experimentos Computacionais}
\label{sec-aestrela-experimentos}
