% ==============================================================================
% TCC - Nome do Aluno
% Capítulo 3 - Especificação de Requisitos
% ==============================================================================
\chapter{Algoritmo A*}
\label{sec-aestrela}

\section{O Algoritmo}
\label{sec-aestrela-algoritmo}
O algoritmo A* (lê-se "A estrela") também conhecido como busca A* é um algoritmo de busca informatizada em grafos. Foi proposta originalmente por (buscar referência) em (buscar referência) e pode ser visto como uma adaptação do algoritmo de Dijkstra (apresentado no capítulo \ref{sec-dijkstra}) em que, ao invés de se calcular a melhor rota de um ponto de partida para todos os demais vértices do grafo, se estabelece uma boa rota (ou mesmo a rota ótima\footnote{A garantia do valor ótimo do algoritmo depende de fatores que serão discutidos na subseção \ref{sec-aestrela-algoritmo-heuristica}.}) partindo do vértice origem a um vértice destino. Isso é feito realizando "podas" do caminho de forma que não seja necessário visitar todos os vértices, apenas os mais promissores.

A seguir é apresentado o algoritmo A* adaptado de \citeonline{russell1995modern,cormen2009introduction}.

\begin{lstlisting}[ mathescape, label=lst-aestrela-codigo, caption=Algoritmo A*., float=htpb]
Entrada: Grafo G, vértice inicial $v_{i}$, vértice destino $v_{f}$
	para todo vértice $s \in V(G), s \neq vi$ faça
		$g(s) = \infty$
	fim para
	$g(v_{i}) = 0$
	para todo vértice $s \in V(G)$ faça
		anterior($s$) = -1
	fimpara
	OPEN = {$v_{i}$}
	enquanto $v_{f}$ não é expandido faça
		s = desenfila(OPEN)
		para cada sucessor $s'$ de $s$ faça
			se $g(s') > g(s) + c(s,s')$ então
				$g(s') = g(s) + c(s,s')$
				anterior($s'$) = s
				insira/atualize $s'$ em OPEN pelo valor de f
			fim se
		fim para
	fim enquanto
retorna anterior
\end{lstlisting}



\subsection{Heurísticas admissíveis e não-admissíveis}
\label{sec-aestrela-algoritmo-heuristica}  

\section{Experimentos Computacionais}
\label{sec-aestrela-experimentos}
