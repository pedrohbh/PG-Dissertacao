% ==============================================================================
% TCC - Nome do Aluno
% Capítulo 5 - Testes Computacionais
% ==============================================================================
\chapter{Considerações Finais e Trabalhos Futuros}
\label{sec-conclusao}

Neste trabalho foram realizados estudos de algoritmos de caminho mínimo para grafos estáticos e dinâmicos, abordando seus funcionamentos, suas estratégias e o impacto que o uso de determinadas estruturas de dados ocasionam. Por meio dos testes computacionais realizados foi possível constatar a eficácia desses algoritmos e analisar em quais situações melhor se aplicam.

Foram estudados os algoritmos de Dijkstra, busca A*, \textit{Anytime Repairing} A* (ARA*) e o \textit{Anytime Dynamic} A* (AD*) sobre instâncias que representam malhas rodoviárias reais, sendo o número de vértices e arestas desses grafos, da ordem de grandeza de 100.000 a 1.000.000, possibilitando assim verificar o desempenho desses algoritmos em uma situação real e verificando também o desempenho para diferentes tamanhos de grafos.

Este trabalho constatou que a escolha de uma determinada estrutura de dados impacta fortemente no desempenho do algoritmo e que certas estrutura de dados propostas, como é o caso da Heap de Fibonacci, apesar de teoricamente possuírem análise computacional melhor do que outras estruturas de dados como a Heap Binária, na prática, nem sempre é essa situação que ocorre, tendo fatores práticos computacionais que prejudicam o desempenho. Foi possível constatar também que o algoritmo A*, em geral, tem um desempenho melhor do que o algoritmo de Dijkstra devido ao uso da estratégia de heurística e de ser objetivo ao se determinar um vértice destino a ser buscado um caminho (em detrimento do Dijkstra que busca a solução calculando a melhor rota para todos os caminhos, sendo seu uso mais recomendado quando a aplicação exige a melhor rota para mais de um vértice). Foi verificado também que o uso de heurísticas não-admissíveis, apesar de se perder a garantia do resultado ótimo, obtêm bons ganhos de tempos computacionais a um preço baixo de deterioramento de qualidade de solução que gira em torno de 4\%. Para os algoritmos dinâmicos, constatou-se que o algoritmo AD* se adéqua muito bem a esse tipo de grafo, tendo que ao se detectar mudança no peso de suas arestas, o algoritmo consegue calcular uma nova rota sem ter que recalcular todos os vértices que antes seriam necessários para achar uma nova rota, já que se baseia no uso de vértices já calculados anteriormente. Já o algoritmo ARA* se mostra muito eficiente para o cálculo rápido de soluções, que não são garantidamente ótimas, mas que a medida que o tempo for passando, essa solução tem a possibilidade de ser melhorada a um custo menor do que se fosse recalculada utilizando todos os vértices necessários.

Para trabalhos futuros sugerimos realizar testes para comparar o desempenho dos algoritmos AD* e ARA* com o D* e o D* Lite, que são algoritmos dinâmicos propostos em \citeonline{stentz1994optimal} e \citeonline{koenig2002d} respectivamente, verificando também em quais situações cada um melhor se aplica. Além disso, vale também um estudo do uso de outras heurísticas além das que foram aplicadas, tanto para o algoritmo A* quanto para o ARA* e o AD*, incluindo também, nestes dois últimos casos, o uso de heurísticas não-admissíveis. Para o algoritmo de Dijkstra sugerimos a realização de testes com outras estruturas de dados propostas que não foram testadas por este trabalho.